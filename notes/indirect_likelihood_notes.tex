\documentclass[b5paper,11pt]{article}

%%%%%%%%%%%%%%%%%%%%%%%%%%%%%%%%%%%%%%%%%%%%%%%%%%%%%%%%%%%%%%%%%%%%%%
%%% packages (and their customizations)
%%%%%%%%%%%%%%%%%%%%%%%%%%%%%%%%%%%%%%%%%%%%%%%%%%%%%%%%%%%%%%%%%%%%%%

%%% spacing and layout
\usepackage[paper=b5paper, margin=3mm]{geometry} % screen reading layout
% \usepackage{setspace}
% \onehalfspacing
\usepackage[subtle]{savetrees}  % a bit more compact
% \usepackage{authblk}            % author and affiliation blocks

%%% graphics and colors
\usepackage[x11names,rgb]{xcolor}
\usepackage{tikz}
\usetikzlibrary{fit}
\usetikzlibrary{arrows.meta}

%%% typesetting
\usepackage[tt=false]{libertine}
\usepackage{booktabs}
\usepackage{dcolumn}
\newcolumntype{P}{D{.}{.}{3}}
\newcolumntype{U}{D{.}{.}{2}}
\newcommand{\rawc}[1]{\multicolumn{1}{c}{#1}}

%%% input and output formats and rendering
\usepackage[utf8]{inputenc}
\usepackage[pdfa]{hyperref}
\usepackage{url}
\urlstyle{same}                 % same font as in text

%%% references 
\usepackage[style=authoryear]{biblatex}
\addbibresource{~/papers/index.bib}

%%% drafting
\usepackage{fixme}
\fxsetup{status=draft, theme=color, layout={inline}}
\renewcommand{\fixmelogo}{\textcolor{black}{\colorbox{Firebrick1}%
    {\textsf{\textbf{FIX}}}}}

%%% math
\usepackage{amsmath}
\usepackage{amsthm}
\usepackage{amsfonts}
\usepackage{mathtools}
\mathtoolsset{showonlyrefs}
\usepackage{gensymb}            % consistent math and text mode

%%% misc
\usepackage{datetime2}

%%%%%%%%%%%%%%%%%%%%%%%%%%%%%%%%%%%%%%%%%%%%%%%%%%%%%%%%%%%%%%%%%%%%%%
%%% general macros
%%%%%%%%%%%%%%%%%%%%%%%%%%%%%%%%%%%%%%%%%%%%%%%%%%%%%%%%%%%%%%%%%%%%%%

\DeclareMathOperator{\mvnormal}{MvNormal}
\DeclareMathOperator{\IID}{IID}
\DeclareMathOperator{\vect}{vec}
\newcommand{\reals}{\mathbb{R}}

\begin{document}

\begin{center}
  {\Large{Notes for \textsl{IndirectLikelihood.jl}}}

  \smallskip

  Tam\'as K.~Papp (\url{tkpapp@gmail.com}), \textsl{\DTMtoday}
\end{center}

\noindent These notes contain writeups of the calculations for the \href{https://github.com/tpapp/IndirectLikelihood.jl}{\textsl{IndirectLikelihood.jl}} Julia package which would be tedious to include in the documentation. There is nothing novel here, these results are well-known.

\section{Multivariate normal}
\label{sec:multivariate-normal}

Consider variables $x_i \in \reals^k$ from a multivariate normal distribution
\begin{equation}
  x_i \sim \mvnormal(\mu, \Sigma)
  \qquad
  i = 1, \dots, n;\ \IID
\end{equation}

\subsection{Unweighted}
\label{sec:unweighted}

First, we calculate the unweighted ML estimate. Define the sample mean and variance
\begin{equation}
  m = \frac1n \sum_i x_i
  \qquad
  S = \frac1n \sum (x_i-m)(x_i-m)'
\end{equation}
The log likelihood of the observations is
\begin{align}
  \label{eq:mvnormal-likelihood}
  \ell(x \mid \mu, \Sigma) &= C_n - \frac{n}{2} \log\det(\Sigma) - \frac12 \sum_i (x_i-\mu)'\Sigma^{-1}(x_i-\mu) =\\
              &= C_n - \frac{n}{2} \log\det(\Sigma) - \frac{n}2 (m-\mu)'\Sigma^{-1}(m-\mu)
                - \frac12 \sum_i \bigl(x_i-m)'\Sigma^{-1}(x_i-m)
\end{align}
where we have used the constant
\begin{equation}
  C_n = -\frac{kn}{2} \log(2\pi)
\end{equation}
Since $\Sigma$ is PSD, $m=\mu$ maximizes $\ell$. Differentiation\footnote{Eg \textcite{minka2000old}.} then shows that the ML estimate for the variance matrix is $\Sigma=S$.

We further manipulate the middle term in \eqref{eq:mvnormal-likelihood} as
\begin{equation}
  \sum_i (x_i-m)' \Sigma^{-1} (x_i-m) = \left(\sum_i (x_i-m)'\otimes(x_i-m)'\right) \vect(\Sigma^{-1}) =
  n \vect(S)'\vect(\Sigma^{-1})
\end{equation}
Thus the log likelihood can be calculated using summary statistics $m$ and $S$ as
\begin{equation}
  \label{eq:mvnormal-ll}
  \ell(m, S \mid \mu, \Sigma) = -\frac{n}{2}\left( k\log(2\pi) + \log\det(\Sigma) +
    \bigl[\vect(S) + \vect((m-\mu)(m-\mu)')\bigr]'\vect(\Sigma^{-1})\right)
\end{equation}

\subsection{Weighted}
\label{sec:weight-mult-norm}

Now assume that we have \emph{frequency weights} $w_i$, and define
\begin{equation}
  W = \sum_i w_i
  \qquad
  m = \frac1W \sum_i w_i x_i
  \qquad
  S = \frac1W \sum_i w_i (x_i - m) (x_i - m)'
\end{equation}
Frequency weights should be interpreted loosely as counts, with each $x_i$ representing $w_i$ observations.

The likelihood of all observations is
\begin{equation}
  \ell(x\mid\mu,\Sigma) = -\frac{Wk}{2}\log(2\pi) - \frac{W}{2}\log\det(\Sigma)
  - \frac12 \sum_i w_i (x_i - \mu)'\Sigma^{-1}(x_i-\mu)
\end{equation}
Again,
\begin{equation}
  \sum_i w_i (x_i - \mu)'\Sigma^{-1}(x_i-\mu) =
  \sum_i w_i (x_i - m)'\Sigma^{-1}(x_i-m) + W (m - \mu)'\Sigma^{-1}(m-\mu)
\end{equation}
which is again minimized by $\mu=m$.

The remaining term is
\begin{equation}
  \sum_i w_i (x_i - m)'\Sigma^{-1}(x_i-m) =
  \left( \sum_i w_i (x_i-m)'\otimes(x_i-m)'\right) \vect(\Sigma^{-1}) =
  W \vect(S)'\vect(\Sigma^{-1})
\end{equation}
Finally, the whole loglikelihood is
\begin{equation}
  \label{eq:mvnormal-ll-weighted}
  \ell(m, S \mid \mu, \Sigma) = -\frac{W}{2}\left( k\log(2\pi) + \log\det(\Sigma) +
    \bigl[\vect(S) + \vect((m-\mu)(m-\mu)')\bigr]'\vect(\Sigma^{-1})\right)
\end{equation}
It is easy to see how~\eqref{eq:mvnormal-ll} and~\eqref{eq:mvnormal-ll-weighted} are similar.

\printbibliography

\end{document}

% Local Variables:
% reftex-default-bibliography: ("~/papers/index.bib")
% End:
